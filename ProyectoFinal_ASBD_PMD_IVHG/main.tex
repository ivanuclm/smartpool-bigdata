\documentclass{article}
\usepackage{titlesec}
\usepackage{hyperref}
\setcounter{secnumdepth}{4}
\usepackage[utf8]{inputenc} % For spanish language
\usepackage[spanish]{babel}
\usepackage[T1]{fontenc}

\usepackage{graphicx} % Need for images
\usepackage{tocloft} % Need for TOC with dots
\usepackage{xcolor}
\usepackage{listings}
\usepackage{pgf,tikz}
\usepackage{float} % Allow [H] in figure for correct placement
\usepackage{picture} % Allow braces in lists
\usepackage{blkarray} % Allow matrix
\usepackage{multicol} % Allow multicolumm
\usepackage{array} % Need for change width of lines in tables
\usepackage{amsmath} % Need for equations
\usepackage[hidelinks]{hyperref} % For index and web links
\usepackage{subfig}
\usepackage[margin=1in]{geometry}
\usepackage{wrapfig}
% \usepackage{subcaption}

\usepackage{amsthm, amssymb, amsfonts, fancyhdr, color, comment, graphicx, environ}
\usepackage{mdframed}
\usepackage{indentfirst}
\usepackage[shortlabels]{enumitem}

% New definitions
\newtoks\logo
\newtoks\authorA

\newtoks\subtitle
\newtoks\degree
\newtoks\year
\newtoks\course
\newtoks\professorA
\newtoks\professorB
% End new definitions

\setcounter{secnumdepth}{2} % Change the depth showed of each section

\renewcommand{\arraystretch}{1.5} % More spaces between rows in tables
\renewcommand{\cftsecleader}{\cftdotfill{\cftdotsep}} % Add dots in TOC


% Redefine \maketitle
\makeatletter
\def\@maketitle{

  \includegraphics[scale=0.7]{\the\logo}

  \begin{center}
    \let \footnote \thanks
    \vfill
    {\Huge \textbf{\@title}\par}
    \vspace{5mm}
    {\Large \textbf{\the\subtitle}\par}
    \vfill
    

  \begin{flushright}
    {\Large
      \begin{tabular}[t]{l l}
        \textbf{\underline{Autor:}}         & \the\authorA 
                                            \vspace{0.2cm} \\ 
        \textbf{\underline{Estudios:}}         & \the\degree \vspace{0.2cm} \\
        \textbf{\underline{Curso:}}         & \the\year \vspace{0.2cm} \\
        \textbf{\underline{Asignatura:}}    & \the\course \vspace{0.2cm} \\
        \textbf{\underline{Fecha:}}         & \@date
      \end{tabular}}
  \end{flushright}

\end{center}}
\par
\makeatother
% End of redefine \maketitle



\definecolor{backgroundColour}{rgb}{0.95,0.95,0.92}

\lstdefinestyle{CStyle}{
    backgroundcolor=\color{backgroundColour},   
    basicstyle=\footnotesize,
    breakatwhitespace=false,         
    breaklines=true,                 
    captionpos=b,                    
    keepspaces=true,                 
    numbers=left,                    
    numbersep=5pt,                  
    showspaces=false,                
    showstringspaces=false,
    showtabs=false,                  
    tabsize=2,
    language=C
}

\graphicspath{{images/}}

\titleformat{\paragraph}
{\normalfont\normalsize\bfseries}{\theparagraph}{1em}{}
\titlespacing*{\paragraph}
{0pt}{3.25ex plus 1ex minus .2ex}{1.5ex plus .2ex}

\title{Proyecto Big Data}
\subtitle{\textit{SmartPool}: Sistema de monitorización y análisis de datos en piscinas públicas}

\logo{uclm_simbolo_logo_color.png}
\authorA{Iván Vicente Hernández García de Mora}
\degree{Máster Universitario en Ingeniería Informática}
\year{2025/26}
\course{Arquitectura de Sistemas Big Data | Procesamiento Masivo de Datos}
\date{29 de enero del 2026}

\titleclass{\subsubsubsection}{straight}[\subsection]

\newcounter{subsubsubsection}[subsubsection]
\renewcommand\thesubsubsubsection{\thesubsubsection.\arabic{subsubsubsection}}
\renewcommand\theparagraph{\thesubsubsubsection.\arabic{paragraph}} % optional; useful if paragraphs are to be numbered

\titleformat{\subsubsubsection}
  {\quad\normalfont\normalsize\series\underline}{\thesubsubsubsection}{1em}{}
\titlespacing*{\subsubsubsection}
{0pt}{3.25ex plus 1ex minus .2ex}{1.5ex plus .2ex}

\makeatletter
\renewcommand\paragraph{\@startsection{paragraph}{5}{\z@}%
  {3.25ex \@plus1ex \@minus.2ex}%
  {-1em}%
  {\normalfont\normalsize\bfseries}}
\renewcommand\subparagraph{\@startsection{subparagraph}{6}{\parindent}%
  {3.25ex \@plus1ex \@minus .2ex}%
  {-1em}%
  {\normalfont\normalsize\bfseries}}
\def\toclevel@subsubsubsection{4}
\def\toclevel@paragraph{5}
\def\toclevel@paragraph{6}
\def\l@subsubsubsection{\@dottedtocline{4}{7em}{4em}}
\def\l@paragraph{\@dottedtocline{5}{10em}{5em}}
\def\l@subparagraph{\@dottedtocline{6}{14em}{6em}}
\makeatother

\setcounter{secnumdepth}{4}

\setcounter{tocdepth}{4}
\begin{document}

\maketitle

\thispagestyle{empty}

% \newpage

% \tableofcontents

\newpage

%%%%%%%%%%%%%%%%%%%%%%%%%%%%%%%%%%%%%%%%%%%%%%%%%%%%%%%%%%%%%%%%%

\section{Introducción y objetivos}
El proyecto \textit{SmartPool} se centra en construir una arquitectura Medallion (Bronze/Silver/Gold) capaz de integrar tres tipos de ingesta: datos estructurados en batch, datos semiestructurados en batch y datos en streaming. 

\vspace{5mm}

La idea principal es disponer de un flujo completo que cubra la captura, limpieza, enriquecimiento y analítica operativa de piscinas: calidad del agua, mantenimiento y consumo energético. 

\vspace{5mm}


Se comenzó trabajando con notebooks (\texttt{notebooks/01..07\_*.ipynb}) para validar cada fase y, una vez estabilizados los resultados, se transformaron en scripts Python ejecutables desde Airflow (\texttt{spark-apps/}).

\vspace{5mm}

El resultado es un sistema reproducible que automatiza ingestas, mantiene histórico en Delta y combina información batch con eventos en tiempo real.

\vspace{5mm}

\section{Diseño Medallion y configuración}
La persistencia se organiza en MinIO con rutas S3A para las capas Bronze/Silver/Gold y una zona \texttt{\_state} para checkpoints de ingesta incremental.

\vspace{5mm}

La configuración de Spark se centraliza en \texttt{spark-apps/smartpool\_config.py} para evitar modificar múltiples archivos, incluyendo parámetros S3A, extensiones de Delta Lake, zona horaria y conexión JDBC a SQL Server. Además, se fijan valores por defecto mediante variables de entorno para facilitar despliegues consistentes. 

\vspace{5mm}

Para mejorar la estabilidad de ejecución, se ajustaron límites de recursos y paralelismo de los jobs lanzados desde Airflow en \texttt{docker-compose.yml}. Esto fue clave para evitar bloqueos por consumo de memoria y para mantener la ejecución concurrente de Airflow y Spark sin saturar el entorno.

\vspace{5mm}

\section{Pipeline 1: Ingesta estructurada batch (SQL Server)}
\begin{itemize}
  \item \textbf{Bronze}: \texttt{02\_ingest\_smartpool.py} lee vía JDBC e implementa ingesta incremental con un checkpoint Delta en \texttt{s3a://spark/medallion/\_state}.
  \item \textbf{Silver}: \texttt{03\_silver\_smartpool.py} limpia y consolida la última versión por clave usando \texttt{updated\_at}.
  \item \textbf{Gold}: \texttt{04\_gold\_smartpool.py} enriquece tareas de mantenimiento con datos de piscinas y calcula una estimación de costes.
\end{itemize}
Las tablas origen son \texttt{pools\_dim} y \texttt{maintenance\_events}, con las que se construye el historial operativo y de mantenimiento sobre el que se apoyan las capas Silver y Gold.

\newpage
\section{Pipeline 2: Ingesta semiestructurada batch (CSV electricidad)}
Los CSV se generan con el productor
\texttt{producer\_electricity\_csv.py} y se guardan en MinIO en \\
\texttt{landing/electricity\_prices/date=YYYY-MM-DD}. El pipeline:
\begin{itemize}
  \item \textbf{Bronze/Silver}: \texttt{05\_ingest\_electricity\_csv.py} normaliza tipos, preserva \texttt{ingest\_date} y particiona por \texttt{date}.
  \item \textbf{Gold}: \texttt{06\_gold\_electricity\_enrichment.py} genera estadísticas diarias y horas pico.
\end{itemize}
Este flujo se centra en disponer de series diarias de precios fiables para cruzarlas después con el consumo estimado de cada piscina.

\vspace{5mm}

\section{Pipeline 3: Ingesta streaming (Kafka)}
El productor \texttt{producer\_smartpool\_sensors.py} envía eventos JSON al tópico \texttt{smartpool-sensors}. El job \texttt{07\_kafka\_smartpool\_sensors.py} aplica:
\begin{itemize}
  \item Lectura de Kafka con \texttt{startingOffsets=latest}.
  \item Parseo de JSON a schema, \textbf{watermark} y ventanas de 1 minuto.
  \item Escritos en Bronze/Silver y agregados Gold por ventana.
  \item Enriquecimiento con \texttt{pools\_dim} (batch) para contexto.
\end{itemize}
El streaming se orienta a disponer de señales casi en tiempo real para generar agregados operativos y alertas tempranas.

\vspace{5mm}

\section{Orquestación en Airflow}
Se han definido tres DAGs:
\begin{itemize}
  \item \textbf{dag\_10\_smartpool\_structured\_batch}: ingesta SQL Server y transformaciones Silver/Gold. Se programa con cron \texttt{0 0 * * *} y lanza el DAG de electricidad con \texttt{TriggerDagRunOperator}.
  \item \textbf{dag\_20\_electricity\_semi\_batch}: pipeline de CSV. Incluye bifurcación con \texttt{BranchPythonOperator} y usa \textbf{XCom} (task \texttt{pick\_ingest\_date}) para pasar la fecha a los jobs Spark.
  \item \textbf{dag\_30\_sensors\_streaming}: streaming Kafka. Usa \texttt{@hourly} como macro temporal y un \\ \texttt{ExternalTaskSensor} para esperar la disponibilidad de datos batch.
\end{itemize}
El encadenamiento entre DAGs y el uso de sensores permite coordinar la llegada de datos batch antes de ejecutar el streaming enriquecido.

\vspace{5mm}

\section{Planificación temporal}
La ejecución batch estructurada se programa diariamente con cron (cada día a las 00:00h). El pipeline de electricidad se activa desde el DAG estructurado para mantener coherencia en la fecha de ingesta. El streaming se ejecuta de forma periódica (macro \texttt{@hourly}) con ventanas de 1 minuto y watermark de 2 minutos, dejando los agregados listos para análisis operativo y alertas.

\section{Retos encontrados y soluciones}
Algunos de los problemas encontrados son:
\begin{itemize}
  \item \textbf{Precisión de timestamps}: en SQL Server los tiempos se almacenaban con 7 decimales, mientras que en Delta se leían con 6, lo que hacía que algunas filas quedaran fuera en filtros incrementales. Se ajustaron conversiones y comparaciones para evitar pérdidas.
  \item \textbf{Ingesta incremental}: la tabla \texttt{pools\_dim} no tenía inicialmente \texttt{updated\_at}, lo que impedía la incrementalidad. Se incorporó ese campo y se habilitó el checkpoint en \texttt{\_state}.
  \item \textbf{Particiones y configuración}: se revisaron particiones y parámetros de Spark para equilibrar rendimiento y consumo de recursos, evitando bloqueos cuando Airflow y Spark coincidían en ejecución.
  \item \textbf{Dependencias Spark (packages/jars)}: al usar \texttt{configure\_spark\_with\_delta\_pip} no se cargaban correctamente los extra jars (por ejemplo el token provider de Kafka), lo que llevó a intentar resolverlo con Ivy. Finalmente se solucionó centralizando jars locales y configuraciones.
  \item \textbf{Recursos en Docker}: se sufrieron caídas y lentitud al ejecutar varios procesos a la vez. Para solucionarlo, se ajustó el paralelismo de Airflow (\texttt{AIRFLOW\_\_CORE\_\_PARALLELISM}, \texttt{MAX\_ACTIVE\_TASKS\_PER\_DAG}, \texttt{MAX\_ACTIVE\_RUNS\_PER\_DAG}) y se limitaron recursos de Spark desde \texttt{docker-compose.yml} (\texttt{SPARK\_EXECUTOR\_MEMORY}, \texttt{SPARK\_EXECUTOR\_CORES}, \texttt{SPARK\_EXECUTOR\_INSTANCES}, \texttt{SPARK\_DRIVER\_MEMORY}, \texttt{SPARK\_CORES\_MAX}).
  \item \textbf{Sincronización entre DAGs}: ajustes en sensores para alinear fechas lógicas y garantizar ejecuciones encadenadas.
\end{itemize}
Estas correcciones permitieron estabilizar los DAGs y asegurar que los tres pipelines funcionasen de forma natural.

\vspace{5mm}

\section{Cumplimiento de requisitos}
\begin{itemize}
  \item \textbf{1.1 Macro temporal}: \texttt{dag\_30} usa \texttt{@hourly}.
  \item \textbf{1.3 Cron}: \texttt{dag\_10} usa \texttt{0 0 * * *}.
  \item \textbf{2.1 Bifurcación}: \texttt{BranchPythonOperator} en \texttt{dag\_20}.
  \item \textbf{2.3 XCom/Taskflow}: \texttt{pick\_ingest\_date} pasa \texttt{ds} a tareas Spark.
  \item \textbf{3.1 Sensor}: \texttt{ExternalTaskSensor} en \texttt{dag\_30}.
  \item \textbf{3.2 Push/Pull entre DAGs}: \texttt{TriggerDagRunOperator} de \texttt{dag\_10} a \texttt{dag\_20}.
  \item \textbf{PMD Medallion}: tres pipelines (batch estructurado, batch CSV y streaming Kafka) con Delta en Bronze/Silver/Gold.
\end{itemize}

\vspace{5mm}

\section{Trabajo futuro}
Como líneas de evolución, se plantea integrar fuentes externas de contexto (AEMET u Open-Meteo) para correlacionar clima y calidad del agua, ampliar el modelo energético con tarifas reales y añadir paneles de visualización y alertas. 

\vspace{5mm}

También se podría incorporar control de calidad automático y modelos predictivos para anticipar incidencias en piscinas con mayor carga de uso.


\end{document}
